\documentclass[12pt,a4paper]{article}

\usepackage[brazil]{babel}
\usepackage[latin1]{inputenc}
\usepackage[T1]{fontenc}

\usepackage{amsmath}
\usepackage{amsfonts}
\usepackage{amssymb}
\usepackage{comment}

\author{Grupo 3}
\title{Gloss\'{a}rio}
\begin{document}

\maketitle

\begin{enumerate}
\item [TCP-IP:] Modelo de redes de computadores que organiza os protocolos n\'{i}veis de abstra\c{c}\~{a}o comumente chamados de camadas.

\item [UDP:] User Datagram Protocol. Protocolo da camada de transporte do modelo TCP-IP respons\'{a}vel por transferir informa\c{c}\~{o}es entre diferentes pontos da rede de forma leve e sem garantia de entrega de dados.

\item [IP:] Internet Protocol. Protocolo da camada de rede do modelo TCP-IP respons\'{a}vel por endere\c{c}ar pontos de uma rede.

\item [SDP:] Session Description Protocol. Protocolo da camada de aplica\c{c}\~{a}o do modelo TCP-IP respons\'{a}vel por descrever sess\~{o}es multim\'{i}dia.

\item [RTP:] Real Time Protocol. Protocolo da camada de aplica\c{c}\~{a}o do modelo TCP-IP respons\'{a}vel por transportar m\'{i}dia de tempo real através de uma rede.

\item [SIP:] Session Initiation Protocol. Protocolo da camada de aplica\c{c}\~{a}o do modelo TCP-IP respons\'{a}vel por iniciar, alterar e terminar sess\~{o}es multim\'{i}dia. SIP utiliza o SDP para negociar as caracter\'{i}sticas das sess\~{o}es que ele gerencia. Neste Trabalho SIP é baseado em UDP.

\item [Softphone:] Software com funcionalidades de um telefone.

\item [Sip Phone:] Softphone baseado em SIP.

\item [Sip Proxy:] Software respons\'{a}vel por redirecionar mensagens SIP. Neste Trabalho um Sip Proxy ou simplesmente Proxy tamb\'{e}m realiza opera\c{c}\~{o}es de autentica\c{c}\~{a}o.

\item [Codec:] Codificador/Decodificador. Algor\'{i}tmo que codifica, compacta e encapsula mídia.

\item [Raw:] Formato Cru. M\'{i}dia sem codifica\c{c}\~{a}o.

\item [Chamador:] Pessoa que inicia uma chamada.

\item [Chamado:] Pessoa que recebe uma chamada.

\end{enumerate}

\end{document}
